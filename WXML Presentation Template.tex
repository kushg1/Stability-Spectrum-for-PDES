% $Header: /cvsroot/latex-beamer/latex-beamer/solutions/conference-talks/conference-ornate-20min.en.tex,v 1.6 2004/10/07 20:53:08 tantau Exp $

\documentclass{beamer}

% This file is a solution template for:

% - Talk at a conference/colloquium.
% - Talk length is about 20min.
% - Style is ornate.



% Copyright 2004 by Till Tantau <tantau@users.sourceforge.net>.
%
% In principle, this file can be redistributed and/or modified under
% the terms of the GNU Public License, version 2.
%
% However, this file is supposed to be a template to be modified
% for your own needs. For this reason, if you use this file as a
% template and not specifically distribute it as part of a another
% package/program, I grant the extra permission to freely copy and
% modify this file as you see fit and even to delete this copyright
% notice. 


\mode<presentation>
{
  \usetheme{CambridgeUS}
  % or ...

  \setbeamercovered{transparent}
  % or whatever (possibly just delete it)
}


\usepackage[english]{babel}
% or whatever

\usepackage{multicol, hyperref}

\usepackage[latin1]{inputenc}
% or whatever

\usepackage{times}
\usepackage[T1]{fontenc}
% Or whatever. Note that the encoding and the font should match. If T1
% does not look nice, try deleting the line with the fontenc.


\title
{Washington Experimental Mathematics Lab} % (optional, use only with long paper titles)

\subtitle
{Project Title Here}

\date{Spring 2017}

% - Give the names in the same order as the appear in the paper.
% - Use the \inst{?} command only if the authors have different
%   affiliation.

\institute[University of Washington] % (optional, but mostly needed)
{
%  \inst{1}%
  Department of Mathematics\\
  University of Washington}
  %\and
  %\inst{2}%
  %Department of Theoretical Philosophy\\
  %University of Elsewhere}
% - Use the \inst command only if there are several affiliations.
% - Keep it simple, no one is interested in your street address.

%\date[CFP 2003] % (optional, should be abbreviation of conference name)
%{ REGS 2011 }
% - Either use conference name or its abbreviation.
% - Not really informative to the audience, more for people (including
%   yourself) who are reading the slides online

% This is only inserted into the PDF information catalog. Can be left
% out. 



% If you have a file called "university-logo-filename.xxx", where xxx
% is a graphic format that can be processed by latex or pdflatex,
% resp., then you can add a logo as follows:



% Delete this, if you do not want the table of contents to pop up at
% the beginning of each subsection:
%\AtBeginSubsection[]
%{
%  \begin{frame}<beamer>
%    \frametitle{Outline}
%    \tableofcontents[currentsection,currentsubsection]
%  \end{frame}
%}


% If you wish to uncover everything in a step-wise fashion, uncomment
% the following command: 

%\beamerdefaultoverlayspecification{<+->}


\begin{document}

\begin{frame}
  \titlepage
\end{frame}




% Structuring a talk is a difficult task and the following structure
% may not be suitable. Here are some rules that apply for this
% solution: 

% - Exactly two or three sections (other than the summary).
% - At *most* three subsections per section.
% - Talk about 30s to 2min per frame. So there should be between about
%   15 and 30 frames, all told.

% - A conference audience is likely to know very little of what you
%   are going to talk about. So *simplify*!
% - In a 20min talk, getting the main ideas across is hard
%   enough. Leave out details, even if it means being less precise than
%   you think necessary.
% - If you omit details that are vital to the proof/implementation,
%   just say so once. Everybody will be happy with that.



%\begin{frame}
 % \frametitle{Make Titles Informative.}

  %You can create overlays\dots
  %\begin{itemize}
  %\item using the \texttt{pause} command:
   % \begin{itemize}
    %\item
     % First item.
      %\pause
    %\item    
      %Second item.
    %\end{itemize}
 % \item
  %  using overlay specifications:
   % \begin{itemize}
    %\item<3->
     % First item.
    %\item<4->
     % Second item.
    %\end{itemize}
  %\item
   % using the general \texttt{uncover} command:
    %\begin{itemize}
     % \uncover<5->{\item
      %  First item.}
      %\uncover<6->{\item
       % Second item.}
    %\end{itemize}
  %\end{itemize}
%\end{frame}

\section{Project Goals}

\begin{frame}{Project TItle}

\begin{description}

\item[Motivation] 

\item[Problem] 

\item[Methods] 


\end{description}


\end{frame}
\section{Reporting}

\begin{frame}{Progress}


\begin{description}

\item[What's worked] 

\item[What hasn't] 

\end{description}

\end{frame}

\begin{frame}{Pictures}

\end{frame}


\section{What's next}



\begin{frame}{Future goals}

\begin{description}

\item[Next steps] 
\item[Challenges] 

\end{description}

\end{frame}



\end{document}


